\documentclass{report}

%\usepackage[latin1]{inputenc}
\usepackage[utf8x]{inputenc}
\usepackage[T1]{fontenc}
\usepackage[francais]{babel}
\usepackage[cm]{fullpage}
\usepackage[pdftex]{graphicx}
\usepackage{setspace}
\usepackage{pgfgantt}
\usepackage{comment} 
\usepackage{amsmath}

\begin{document}

%page de garde
\begin{titlepage}

%logo de la fds, de l'UM et des infos
\includegraphics[scale=0.5]{logoFDS.png}
\hfill
\includegraphics[scale=0.2]{logoInfo.jpg}
\vspace{1cm}

\begin{center}
%au dessus du titre
\textsc{\Large{Rapport de projet T.E.R}} \\
\vspace{1cm}
\textsc{\Large{Projet Informatique HLIN405}} 
\vspace{1.5cm}

%titre
\doublespacing{\textsc{\huge{PunyDuck}}} \\
\vspace{2cm}
\includegraphics[scale=0.5]{logoPunyDuck.jpg}
\vfill
\end{center}

%noms/prenoms + Encadrante
\begin{minipage}[t]{8.5cm}
	\begin{flushleft}
	    \large{\textbf{Etudiants :}}
	    \begin{itemize}
	        \item \large{Valentin \bsc{FONTAINE}}
	        \item \large{Paul  \bsc{BUNEL}} 
	        \item \large{Esteban \bsc{BARON}}
	        \item \large{Valentin \bsc{PERON}}
	        \item \large{Julien \bsc{LEBARON}}
	    \end{itemize}
		\vspace{0.5cm}
		\large{\textbf{Encadrante :}}
		\large{Anne-Elisabeth \bsc{BAERT}} \\
	\end{flushleft}
\end{minipage}
\hfill
%année universitaire
\begin{minipage}[t]{8cm}
	\begin{flushright} 
		\large{\textbf{Année :}} 
		\large{2019-2020}
	\end{flushright}
\end{minipage}
\end{titlepage}

%Sommaire
\begin{titlepage}
\renewcommand{\contentsname}{Sommaire}
\large{\tableofcontents}
\thispagestyle{empty}
\end{titlepage}


\renewcommand{\chaptername}{Partie}

\chapter{Introduction}
    \section{motivation du projet (pourquoi on l'a choisi)}
    \section{cahier des charges}
    \section{def de mot complexes peut etre}

\chapter{Organisation}
    \section{répartition des tâches}
    \section{méthode de travail}
    \section{outils utilisés}
\chapter{Conception}
\section{étapes de conception(genre paragraphe qui fait office de sommaire)}
\section{description de ces étapes}

\chapter{Implémentation}
Préciser les difficultés rencontrées pour chaques groupes ou parties. ET dire ce qu'on a fait pour les résoudres


\chapter{Bilan}
    \section{Difficulté rencontré (les plus grosses ou celles qui ont touché plusieurs groupes)}
    \section{Bilan en se reportant sur le cahiers des charges}
    Qu'est-ce qui a été fait et pas fait. Des modifications? Des Ajouts?
    
\chapter{Conclusion}

\chapter{Annexe}
    \section{diagramme de gantt}
    \section{copie du code}

\end{document}
