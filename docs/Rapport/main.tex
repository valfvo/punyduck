\documentclass{report}

%\usepackage[latin1]{inputenc}
\usepackage[utf8x]{inputenc}
\usepackage[T1]{fontenc}
\usepackage[francais]{babel}
\usepackage[cm]{fullpage}
\usepackage[pdftex]{graphicx}
\usepackage{setspace}
\usepackage{pgfgantt}
\usepackage{comment} 
\usepackage{amsmath}

\begin{document}

%page de garde
\begin{titlepage}

%logo de la fds, de l'UM et des infos
\includegraphics[scale=0.5]{logoFDS.png}
\hfill
\includegraphics[scale=0.2]{logoInfo.jpg}
\vspace{1cm}

\begin{center}
%au dessus du titre
\textsc{\Large{Rapport de projet T.E.R}} \\
\vspace{1cm}
\textsc{\Large{Projet Informatique HLIN405}} 
\vspace{1.5cm}

%titre
\doublespacing{\textsc{\huge{PunyDuck}}} \\
\vspace{2cm}
\includegraphics[scale=0.5]{logoPunyDuck.jpg}
\vfill
\end{center}

%noms/prenoms + Encadrante
\begin{minipage}[t]{8.5cm}
	\begin{flushleft}
	    \large{\textbf{Etudiants :}}
	    \begin{itemize}
	        \item \large{Valentin \bsc{FONTAINE}}
	        \item \large{Paul  \bsc{BUNEL}} 
	        \item \large{Esteban \bsc{BARON}}
	        \item \large{Valentin \bsc{PERON}}
	        \item \large{Julien \bsc{LEBARON}}
	    \end{itemize}
		\vspace{0.5cm}
		\large{\textbf{Encadrante :}}
		\large{Anne-Elisabeth \bsc{BAERT}} \\
	\end{flushleft}
\end{minipage}
\hfill
%année universitaire
\begin{minipage}[t]{8cm}
	\begin{flushright} 
		\large{\textbf{Année :}} 
		\large{2019-2020}
	\end{flushright}
\end{minipage}
\end{titlepage}

%Sommaire
\begin{titlepage}
\renewcommand{\contentsname}{Sommaire}
\large{\tableofcontents}
\thispagestyle{empty}
\end{titlepage}

%renommer les chapitres en parties
\renewcommand{\chaptername}{Partie}



%Introduction 1/2 pages
\chapter{Introduction} %Partie présentation
Dans le cadre du TER de notre deuxième année à la faculté des sciences de Montpellier nous avons proposé un projet s'intitulant PunyDuck. C'est une plate-forme de distribution des projets des étudiants du département informatique.\\

Le groupe de développement est composé de cinq personnes, Valentin \textsc{FONTAINE}, Paul \bsc{BUNEL}, Valentin \bsc{PERON}, Julien \bsc{LEBARON} et Esteban \bsc{BARON}. Nous sommes encadré par Mme Anne-Elisabeth \bsc{BAERT}.

\vspace{1cm}
\textbf{\huge{}{Motivation}}\\

Le TER est un module qui apporte beaucoup aux étudiants en gestion de projet ainsi qu'en programmation. Seulement une fois terminés les projets ne sont pas valorisés et tombent dans l'oubli. Notre solution est de proposer une application permettant à chaque étudiants de déposer leurs projets pour les rendre visibles et téléchargeables par tous.

\vspace{1cm}
\textbf{\huge{}{Approches}}\\ %Esteeeeeeeeeeee petit passage en plus

Les différentes approches faces a notre problématique

\vspace{1cm}
\textbf{\huge{}{Cahier des charges}}\\

Objectifs : Créer une plateforme de distribution des projets des étudiants de la Faculté des
sciences de l'Université de Montpellier.\\
Différentes étapes\\
\begin{itemize}
    \item Mise en place d’un serveur qui servira d’intermédiaire entre les utilisateurs et la
base de données.
    \item Création d’un framework pour faciliter la réalisation de l’application.
    \item Conception de l’application graphique à l’aide du framework.
    \item Connexion entre l’application et le serveur.
    \item Création d’une base de données pour stocker les comptes des utilisateurs et les
projets.
    \item Mise en service d’un site internet permettant le téléchargement de l’application.
Cahier des charges
\end{itemize}
I - Le serveur
\begin{itemize}
    \item Fonctionnement asynchrone
    \item Héberger de manière sécurisée les données des utilisateurs.
    \item Les utilisateurs pourront télécharger les projets hébergés.
    \item Doit pouvoir gérer la plupart des erreurs de réseau, comme la coupure de la
    connexion lors d’un téléchargement.
\end{itemize}
II - Le framework
\begin{itemize}
    \item Ensemble de classes et fonctions qui serviront de base à la structure d’une nouvelle
application.
    \item Son utilisation doit être facile avec assez de fonctionnalités déjà disponibles pour
    pouvoir réduire les appels de fonctions bas niveau.
    \item Permettre la plasticité des interfaces
\end{itemize}
III - L’application
\begin{itemize}
    \item Interface graphique permettant de naviguer facilement entre différents onglets
Possibilité d'accéder aux informations de son compte et de les modifier :
pseudonyme, adresse mail, âge, niveau, badges.
    \item Personnalisation limitée : mode clair / sombre, position de certains éléments de la
fenêtre, couleur de l’arrière plan.
\end{itemize}
IV - La base de données
\begin{itemize}
    \item Stocke les données relatives aux comptes des utilisateurs ainsi que les projets.
Les données seront cryptées.
\end{itemize}
V - Le site
\begin{itemize}
    \item Une page pour télécharger l’application.
    \item Design responsive et dynamique
\end{itemize}

%Partie 2 Partie technologies utilisées 1/2 pages
\chapter{Technologies utilisées}
\section{Langages}
Le choix du langage pour coder l'application a été le C/C++ pour sa fiabilité, son efficacité, et sa très grande possibilité d'utilisation. De plus le C++ nous a permis de coder en orienté objet, ainsi permettant la réalisation de l'application.
Dans le cadre des UEs HLIN202 et HLIN302 basé sur le langage C++, cela nous a permis de partir sur ce choix de langage en étant rassurés. \\ % à reformuler 

Pour gérer la partie réseau de notre application, nous n'avions pas besoin d'utiliser un langage puissant comme le C/C++, nous pouvions donc nous rabattre sur un langage moins performant mais plus facile d'accès.
De ce fait, nous nous sommes dirigés vers le langage python, qui est un langage très simple d'utilisation, pratique pour débuter dans le domaine complexe de la programmation que représente la programmation en réseau. De surcroît il est plutôt efficace pour gérer les entrées sorties, notamment pour la lecture et l'écriture dans des fichiers, qui est une notion centrale dans notre projet.\\

Cependant, le langage Python de base ne permet pas de faire de la programmation en réseau, nous avons donc du choisir une des nombreuses bibliothèques disponibles permettant de faire de la programmation réseau en Python. Après réflexion, nous avons opté pour le module « asyncio » pour deux raisons : premièrement, ce module offrait une interface de programmation en réseau de haut niveau donc plus simple d'utilisation, ce qui nous arrangeait particulièrement étant donné que nous sommes parfaitement novices dans ce domaine ; deuxièmement, le gros point fort de ce module est le fait qu'il implémente une nouvelle manière de programmer~: la programmation asynchrone. La programmation asynchrone pour les entrées/sorties est une forme de programmation parallèle permettant d'exécuter d'autres parties d'un programme lorsque celui-ci est en attente d'une transmission de donnée, afin de grandement diminuer le temps d'exécution du programme.
\section{Outils} %outils utilisés pourquoi ce choix avantages ? postgresql

%Partie 3 Développement logiciel 5/10 pages
\chapter{Développement logiciel, Conception, Modélisation, Implémentation}
\section{Présentation du développement}
\section{Modélisation} %UML
\section{Fonctionnalités de l'interface}
%Je parle de ce qui devait être fait (j'ai pas mis les notification, ni personnalisation) si quelque chose ne peux être fait ou n'est pas fait d'ici la fin, enlever directement la partie en question.
L'interface graphique se décompose de plusieurs fenêtres chacune spécifique d'autant plus que l'ensemble peux être personnalisable (placement de certains bloc présent sur l'interface). Pour la partie commune à chaque fenêtre de l'interface
on a la barre des onglets contenant, dans cet ordre :
\vspace{0.5cm}
\begin{itemize}
    \item "L'accueil", comprenant les lancements rapide d'application dernièrement utilisées ainsi que les nouveautés lié à l'application ou de certains projets.
    \item Les "Projets", où l'ensemble des projets mis en ligne seront affichés. 
    \item La "Collection", contenant les projets suivie et télécharger seront directement affichés.
    \item Le "Profil" est l'endroit où l'on peut modifier ses données personnelles, sa page de profil et les paramètres de l'application.
    \item Les "Contact", seront les différentes personnes suivie ou qui nous suive avec la possibilités d'échanger par message dans l'application.
    \item La "Communauté" recensera l'ensemble des forums d'aides.
    \item Le "Dépôt" vas servir à faire une demande de dépôt de projet, afin qu'il soit vérifier avant mise en ligne (pour éviter les dépôt de virus ou autre).
    \item Icônes basique tel que épinglé, réduire et fermer. Mais contient aussi le mode nuit (fond clair qui devient foncé).
\end{itemize}
\vspace{0.5cm}
La fonctionnalité principale étant d'aller sur la page projet, voir un qui nous intéresse ou faire une demande de dépôt pour notre projet. Maintenant cas par cas voyons les différentes interactions.\\
Pour l'accueil il est possible de cliquer sur les articles mis en avant enfin de pouvoir les lires en détails, mais aussi les lancements rapides des derniers projet lancer en cliquant sur l'icône en question. Pour les projets, les fonctionnalitées qui changes sont les touches de tri et d'affichage ainsi que la barre rechercher. Pour la collection, c'est pareil sauf que l'on peux acceder à un menu d'action a coté de chaque projet (ressemblant à 3 petit points):
\begin{itemize}
    \item Désinstaller
    \item Emplacement
    \item Page projet
    \item Désabonner / Ne plus suivre
\end{itemize}
Ainsi que les boutons d'action, lancer qui vas démarrer l'application, et installer qui vas télécharger et installer l'application.\\
En ce qui concerne la page de présentation du projet, on peux interagir avec la notations (données une notes ou envoyé un commentaire) mais également un lien qui va envoyé vers un forum existant en lien avec le projet. On a notamment accès au bouton suivre pour l'avoir dans notre collection et le bouton j'aime pour aimé un projet.\\
Les interactions avec l'interface concernant le profil sont, la modification de fond et de l'image de profil ainsi que des autres paramètres (nom, prénom, mot de passe,...), l'accès au paramètre du client (langue,emplacement,...) et le bouton de déconnexion. Pour l'interface communauté, on peux trié les différents forum (aide, suggestion, bugs, autres) ou encore en rechercher un. De plus on a le bouton d'ajout de forum.


\section{Format des données et procédure d'utilisation} % dnnées utilisées ou encore convention, décrire certaines procédures de lecture et de validation
\section{Statistique} %nombres de classes/scripts/ lignes de code/ nombre de module

%Partie 4 
\chapter{Algorithmes et Structures de Données}
\section{Présentation des principales structures de données}
\section{Présentation des principaux algorithmes}%présentation et description de deux ou un algo très important et intéressant 
\section{Complexité théorique}

%.Partie 5 1/2 pages
\chapter{Gestion du Projet}
\section{Organisation et planification} 
%gantt en latex, les [progress=0] ne sont pas obligatoire. Il faut rassembler plusieurs commit en un.
\begin{comment} %ça met en commentaire
\definecolor{bargreen}{RGB}{133,193,132}
\definecolor{groupgreen}{RGB}{53,107,52}
\definecolor{darkgreen}{RGB}{35,68,35}
\definecolor{linkred}{RGB}{165,0,33} %rouge

%police du gantt au choix selon préférence
%\renewcommand\sfdefault{phv}
%\renewcommand\mddefault{mc}
%\renewcommand\bfdefault{bc}

\setganttlinklabel{s-s}{START-TO-START}
\setganttlinklabel{f-s}{FINISH-TO-START}
\setganttlinklabel{f-f}{FINISH-TO-FINISH}
\sffamily
\begin{ganttchart}[
    canvas/.append style={fill=none, draw=black!25, line width=3pt},
    hgrid style/.style={draw=black!5, line width=.75pt},
    vgrid={*1{draw=black!5, line width=.75pt}},
    today label font=\small\bfseries,
    title/.style={draw=none, fill=none},
    title label font=\bfseries\footnotesize,
    title label node/.append style={below=7pt},
    include title in canvas=false,
    %Zone des sous parties d'un groupe
    bar label font=\mdseries\small\color{black!90},
    bar label node/.append style={left=1.8cm},
    bar/.append style={draw=none, fill=darkgreen!60},
    bar incomplete/.append style={fill=bargreen},
    bar progress label font=\mdseries\footnotesize\color{black!70},
    %tête de groupe (Réseau, framwork)
    group incomplete/.append style={fill=groupgreen!85},
    group/.append style={draw=none, fill=darkgreen},
    group left shift=0,
    group right shift=0,
    group height=0.3,
    group peaks tip position=0,
    group label node/.append style={left=2.5cm},
    group progress label font=\bfseries\small,
    %zone rouge / avancement
    link/.style={-latex, line width=1.5pt, linkred},
    link label font=\scriptsize\bfseries,
    link label node/.append style={below left=-2pt and 0pt}
  ]{1}{15}
  \gantttitle[
    title label node/.append style={below left=7pt and -3pt}
  ]{SEMAINES:\quad1}{1}
  \gantttitlelist{2,...,15}{1} \\
  \ganttgroup{Partie Réseaux}{1}{15} \\
  \ganttbar[progress=75,name=PR1A]{\textbf{PR 1.1} Activité A}{1}{8} \\
  \ganttbar[progress=67,name=PR1B]{\textbf{PR 1.2} Activité B}{1}{3} \\
  \ganttbar[progress=50,name=PR1C]{\textbf{PR 1.3} Activité C}{4}{10} \\
  \ganttbar[progress=0,name=PR1D]{\textbf{PR 1.4} Activité D}{4}{10} \\[grid]
  \ganttgroup{Partie framwork}{1}{25} \\
  \ganttbar[progress=0]{\textbf{PFRA 2.1} Activité E}{4}{5} \\
  \ganttbar[progress=0]{\textbf{PFRA 2.2} Activité F}{6}{8} \\
  \ganttbar[progress=0]{\textbf{PFRA 2.3} Activité G}{9}{10}\\[grid]
  \ganttgroup{Partie Rapport}{15}{25} \\
  \ganttbar[progress=0]{\textbf{PRA 2.1} Activité K}{4}{5} \\
  \ganttbar[progress=0]{\textbf{PRA 2.2} Activité L}{6}{8} 
  
  %link entre activitées
  \ganttlink[link type=s-s]{PR1A}{PR1B}
  \ganttlink[link type=f-s]{PR1B}{PR1C}
  \ganttlink[
    link type=f-f,
    link label node/.append style=left
  ]{PR1C}{PR1D}
\end{ganttchart}
\end{comment}

%\begin{comment} %ça met en commentaire
\definecolor{bargreen}{RGB}{133,193,132}
\definecolor{groupgreen}{RGB}{53,107,52}
\definecolor{darkgreen}{RGB}{35,68,35}
\definecolor{linkred}{RGB}{165,0,33} %rouge

%police du gantt au choix selon préférence
%\renewcommand\sfdefault{phv}
%\renewcommand\mddefault{mc}
%\renewcommand\bfdefault{bc}

\setganttlinklabel{s-s}{START-TO-START}
\setganttlinklabel{f-s}{FINISH-TO-START}
\setganttlinklabel{f-f}{FINISH-TO-FINISH}
\sffamily
\begin{ganttchart}[
    canvas/.append style={fill=none, draw=black!25, line width=3pt},
    hgrid style/.style={draw=black!5, line width=.75pt},
    vgrid={*1{draw=black!5, line width=.75pt}},
    today label font=\small\bfseries,
    title/.style={draw=none, fill=none},
    title label font=\bfseries\footnotesize,
    title label node/.append style={below=7pt},
    include title in canvas=false,
    %Zone des sous parties d'un groupe
    bar label font=\mdseries\small\color{black!90},
    bar label node/.append style={left=1.8cm},
    bar/.append style={draw=none, fill=darkgreen!60},
    bar incomplete/.append style={fill=bargreen},
    bar progress label font=\mdseries\footnotesize\color{black!70},
    %tête de groupe (Réseau, framwork)
    group incomplete/.append style={fill=groupgreen!85},
    group/.append style={draw=none, fill=darkgreen},
    group left shift=0,
    group right shift=0,
    group height=0.3,
    group peaks tip position=0,
    group label node/.append style={left=2.5cm},
    group progress label font=\bfseries\small,
    %zone rouge / avancement
    link/.style={-latex, line width=1.5pt, linkred},
    link label font=\scriptsize\bfseries,
    link label node/.append style={below left=-2pt and 0pt}
  ]{1}{30}
  \gantttitle[
    title label node/.append style={below left=7pt and -3pt}
  ]{SEMAINES:\quad1}{1}
  \gantttitlelist{2,...,30}{1} \\
  \ganttgroup{Préliminaire Projet}{1}{1} \\[grid]
  \ganttgroup[name = R]{Partie Réseaux}{2}{10} \\
  \ganttbar[progress=75,name=PR1A]{\textbf{PR 1.1} Activité A}{2}{8} \\
  \ganttbar[progress=67,name=PR1B]{\textbf{PR 1.2} Activité B}{2}{10} \\
  \ganttbar[progress=0 ,name=PR1C]{\textbf{PR 1.4} Activité C}{4}{10} \\[grid]
  \ganttgroup{Partie framwork}{2}{10} \\
  \ganttbar[progress=0]{\textbf{PFRA 2.1} Activité D}{2}{5} \\
  \ganttbar[progress=0]{\textbf{PFRA 2.2} Activité E}{3}{5} \\
  \ganttbar[progress=0]{\textbf{PFRA 2.3} Activité F}{4}{5}\\[grid]
  \ganttgroup[name= F]{Partie frontwork}{11}{14} \\
  \ganttbar[progress=0]{\textbf{PFRO 2.1} Activité G}{5}{6} \\
  \ganttbar[progress=0]{\textbf{PFRO 2.2} Activité H}{6}{8} \\
  \ganttbar[progress=0]{\textbf{PFRO 2.3} Activité I}{9}{10}\\[grid]
  \ganttgroup{Partie Rapport}{2}{15}\\[grid]
  \ganttgroup{Partie Site web}{9}{13}
  
  %link entre activitées
  \ganttlink[link type=s-s]{PR1A}{PR1B}
  \ganttlink[link type=f-s]{R}{F}
  \ganttlink[link type=f-f,link label node/.append style=left]{PR1C}{PR1B}
\end{ganttchart}
%\end{comment}
\section{Changements majeurs}%ceux effectués en cours de projet (changement d'idée, revoir un algo,...)
Lorem ipsum dolor sit amet, consectetur adipiscing elit. Suspendisse eget augue accumsan nisi luctus gravida. Nulla non risus risus. Suspendisse orci ligula, euismod et ornare eu, suscipit vitae nisi. Morbi massa lacus, rutrum sed sodales sed, luctus vel mi. Integer sed purus lorem. Vivamus nec suscipit sem, nec ultrices tellus. Donec efficitur elit sit amet ultricies vestibulum. Nullam a leo eu enim iaculis feugiat eget id purus. Mauris sed hendrerit quam, vitae ultricies libero. Pellentesque viverra sed ex at dapibus. Aliquam sed auctor dolor, vehicula maximus quam. Vestibulum sed eros eu justo tempor varius. Cras efficitur metus ac iaculis volutpat. 

%.Partie 6 1 page
\chapter{Bilan et Perspectives} %bilan et Conclusion, parler en onction du cahier des charges, les perspectives futur du projet et l'apport.
Le bilan de cette fin de projet est très positive, en comparant à notre cahier des charges, la quasi totalité des demandes et des objectif a été atteints. Malgré le manque de mains d'oeuvres pour la partie informatique, 3 au lieu de 5, le projet à pu voir le jour avec succès.
Bilan partie réseau
Bilan partie front et framwork
Bilan des aides (cours, site, grâce à quoi, etc...)

\vspace{1cm}
\textbf{\huge{}{Conclusion}}\\

Conclusion global, point negatif et positif plus terminer par les perspectives du client.

%.Partie 7
\chapter{Bibliographie et Annexes}

\end{document}
